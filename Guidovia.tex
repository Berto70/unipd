\documentclass{aastex63}
\usepackage[T1]{fontenc}
\usepackage[UTF8]{inputenc}
\usepackage{graphicx}
\usepackage{longtable}
\usepackage{wrapfig}
\usepackage{latexsym}
\usepackage{amssymb}
\usepackage{epsf}
\usepackage{hyperref}
\usepackage[italian,english]{varioref}
\usepackage[greek,italian]{babel}
\usepackage[suftesi]{frontespizio}




\shorttitle{Verifica dell'accelerazione gravitazionale a Padova}
\shortauthors{Verifica della legge di Hooke}

\begin{document}
\title{Verifica dell'accelerazione gravitazionale a Padova}

\author{Gabriele Bertinelli}
\collaboration{1}{1219907}  

\author{Roben Bhatti}
\collaboration{1}{1216914}

\author{Alberto Brusegan}
\collaboration{1}{1230215}

\email{gabriele.bertinelli@studenti.unipd.it\\ roben.bhatti@studenti.unipd.it\\ alberto.brusegan@studenti.unipd.it}
\affiliation{Dipartimento Fisica e Astronomia, UniPD\\
Corso di Astronomia, a.a. 2019-2020}

\section{Introduzione}
La forza agente su un corpo che scivola (senza attrito) su di un piano, inclinato di un angolo $\alpha$ rispetto all'orizzontale, di cui forniamo una schematizzazione 
nella Figura(\ref{fig:schema}), è la forza peso $P=m\,g$; questa è scomposta nelle due componenti:
\begin{itemize}
\item la componente normale al piano, in modulo $m\,g\,\cos\alpha$, che viene equilibrata dalla reazione vincolare $N$; 
\item la componente parallela al piano, in modulo $m\,g\,\sin\alpha=costante$. Essendo costante il moto del corpo 
è uniformemente accelerato, con accelerazione $a=g\,\sin\alpha$; velocità e spazio percorso sono dati da $v(t)=v_0+a\,t$ e $s(t)=s_0+v_0\,t+\frac{1}{2}a\,t^2$.
\end{itemize}
Si può notare che la velocita è proporzionale al tempo e il coefficiente angolare della retta, ovvero l'accelerazione, è proporzionale al seno dell'angolo di
inclinazione del piano ed indipendente dalla massa della slitta.
Dunque l'accelerazione di gravità $g$ si ricava da $g=a/\sin\alpha$.

\begin{figure}
    \centering
    \includegraphics[width=0.75\textwidth]{schema.jpg}
    \caption{Rappresentazione schematica delle forze agenti su di un corpo che scivola, in assenza di attrito, su di un piano.}
    \label{fig:schema}
\end{figure}

\section{Obiettivi}
Verifica del moto rettilineo uniformemente accelerato su un piano inclinato, stima del valore dell'accelerazione di gravità $g$ e verifica della compatibilità del 
valore sperimentale con quello noto calcolato a Padova.

\section{Dettagli tecnici sull'apparato sperimentale}
Come piano inclinato viene utilizzata una rotaia a cuscino d'aria, capace di rendere trascurabile l'attrito tra il corpo e il piano.
La rotaia è composta da un lungo tubo di metallo cavo, all'interno del quale viene soffiata aria da un compressore; l'aria fuoriesce 
da dei piccoli fori posti sulla parte superiore del tubo. La sua funzione di sollevare di pochi millimetri 
un apposito corpo chiamato carrello posto sopra la guidovia, affinché l'attrito sia con buona approssimazione trascurabile. 
La raccolta dati avviene tramite due fotocellule ad infrarossi poste lungo la guidovia: la prima posta sd una distanza di $40\,cm$ dal zona di distacco del carrello, per minimizzare
eventuali effetti generati dall'interferenza del magnete del meccanismo di sgancio e il carrello stesso.\\
La seconda fotocellula è mobile e la distanza dalla prima può essere letta su una scala graduata con incertezza sulla misura di $1\cdot 10^{-3}\,m$.\\
L'incertezza sulla singola misura del cronometro, collegato alle due fotocellule, è di $1\cdot 10^{-4}\,s$.\\
\'E possibile inclinare la guidovia tramite una vite, posta alla base dello strumento; un giro della vite corrisponde a $5'$ e l'incertezza è $5'/24$ o $6.06\cdot 10^{-5}\,rad$.

\section{Raccolta dati}\label{sec:dat}
La raccolta dati è stata effettuata come segue, mentre i dati raccolti sono riportati nell'Appendice \ref{sec:tabracdat}.
\begin{enumerate}
    \item Si procede a trovare la posizione della vite per cui la guidovia è orizzontale: si ruota la vite finché la slitta non rimane immobile 
    una volta lasciata libera. 
    Questa operazione si ripete tre volte: al centro e alle due estremità della guidovia (a circa $40\,cm$ e circa $120\,cm$). Poi la vite si regola in una 
    posizione intermedia tra tutte quelle trovate.
    \item Si inclina la guidovia a $15'$ e si utilizza la slitta scarica. Si pone la prima fotocellula alla posizione $x_i=40\,cm$ mentre la seconda a $x_f=50\,cm$.
    \item Si effettuano cinque misure ripetute di intervalli di tempo impiegati dalla slitta per percorrere lo spazio $x_f-x_i$.
    \item Si ripete il processo spostando la seconda fotocellula ogni volta di $10\,cm$ arrivando fino a $130\,cm$.
    \item Si calcola il valor medio della velocità di percorrenza per ogni tratto di $10\,cm$ e si ricorda che tale velocità coincide con la velocità istantanea 
    nell'istante intermedio $t_{12}$.
    \item Si riportano in grafico i punti $(t_{12},\,v(x_1,x_2))$ e si verifica che si distribuiscano intorno ad una retta.
    \item Si calcolano i parametri $a$ e $b$ della retta interpolante $v=a+b\,t$, con il relativo errore.
    \item Si stima il valore del modulo dell'accelerazione di gravità $g$ a partire dalla pendenza della retta di interpolazione e si associa a tale stima 
    l'errore calcolato con la formula di propagazione degli errori.
    \item Si ripetono i punti 2-8 per l'inclinazione della guidovia pari a $30'$, $45'$, $45'$ con slitta carica.
\end{enumerate}
   
\section{Elaborazione dati}
Come illustrato nella Tabella (\ref{tab:racdat}) si è calcolato $\bar{t}$, tempo medio, con (\ref{eq:1}) ovvero la media aritmetica tra le 5 misure effettuate per ogni tratta e per ogni inclinazione. 
L'errore associato è stato ricavato tramite la (\ref{eq:2}), cioè la deviazione standard corretta (al denominatore $N-1$ invece di $N$). Si è scelto di utilizzare 
questa correzione per ovviare la tendenza della formula "classica" a sottostimare le incertezze, soprattutto nel caso in cui si lavori con pochi dati, come quello che stiamo analizzando.
\begin{equation}
    \bar{t}=\frac{\sum_{n=1}^{N}t_i}{N}
    \label{eq:1}
\end{equation}
\begin{equation}
    \sigma_{\bar{t}}=\sqrt{\frac{\sum_{n=1}^{N}(t_i-\bar{t})^2}{N-1}}
    \label{eq:2}
\end{equation}

Si è poi calcolato $\Delta t$ con la (\ref{eq:3}) ovvero la differenza tra $\bar{t}$ relativo a quella tratta e il precedente. 
Come errore su $\Delta t$ si è usata la (\ref{eq:4}), somma in quadratura.
\begin{equation}
    \Delta t={\bar{t}_i-\bar{t}_{i-1}}
    \label{eq:3}
\end{equation}
\begin{equation}
    \sigma_{\Delta t}=\sqrt{\sigma _{\bar{t}_i}^2+\sigma _{\bar{t}_{i-1}}^2}
    \label{eq:4}
\end{equation}

Si è calcolato il $t_{int}$, tempo intermedio, ovvero l'istante in cui la velocità istantanea è uguale alla velocità calcolata nel $\Delta t$. 
Esso viene calcolato con la (\ref{eq:5}), cioè con media aritmetica tra $\bar{t}_i$ e $\bar{t}_{i-1}$ e l'errore associato si calcola 
con la (\ref{eq:6}) ovvero con la somma in quadratura moltiplicata per un fattore di $\frac{1}{2}$.
\begin{equation}
      t_{int}=\frac{\bar{t}_i+\bar{t}_{i-1}}{2}
      \label{eq:5}
\end{equation}
\begin{equation}
    \sigma _{t_{int}}=\frac{1}{2}\sqrt{\sigma_{\bar{t}_i}^2+\sigma _{\bar{t}_{i-1}}^2}
    \label{eq:6}
\end{equation}

In seguito si è calcolata la velocità con la (\ref{eq:7}), rapportando la lunghezza del tratto $(10\,cm)$ a $\Delta t$.
L'errore associato si ricava tramite la (\ref{eq:8}), formula generale della propagazione degli errori.
\begin{equation}
    v=\frac{0.1}{\Delta t_{i}}
    \label{eq:7}
\end{equation}
\begin{equation}
    \sigma _{v}=\sqrt{\frac{\partial v^2 }{\partial x^2}\sigma _x^2+\frac{\partial v^2}{\partial \Delta t^2}\sigma _t^2}=\sqrt{{\bigg(\frac{1}{\Delta t}\sigma_{x}}\biggr)^2+\biggl(-\frac{x}{\Delta t^2}\sigma_{\Delta t}\biggr)^2}
    \label{eq:8}
\end{equation}

Nella Figura (\ref{fig:graf}) si sono graficati i punti $(t_{12}, v(x_1,\,x_2))$ e si è ricavata la retta interpolante $v=b+a\,t$ e i relativi coefficienti con il metodo dei minimi quadrati.\\
Il parametro $a$, che indica la pendenza della retta, è l'accelerazione, in quanto derivata della velocità. Si ricava l'errore sull'accelerazione con il metodo dei minimi 
quadrati, dove $\sigma_v$ è dato dalla media dei singoli $\sigma_{v_i}$.\\
Si nota come previsto che il moto è uniformemente accelerato e che la massa non influenza l'accelerazione; infatti 
i due fit, dei dati raccolti con inclinazione pari a $45'$ sono sovrapposti. L'incertezza sulla velocità è predominante rispetto all'incertezza sul tempo, infatti l'incertezza sull'ascissa è delle stesse 
dimensioni del punto sul grafico quindi non è apprezzabile. %#inserire nota sotto il grafico%

\begin{equation}
    \sigma _{a}=\sigma_v\sqrt{\frac{N}{\Delta}}=\sigma_v\sqrt{\frac{1}{\sum t_i^2-(\sum t_i)^2}}
    \label{eq:9}
\end{equation}

Nella Tabella (\ref{tab:comp}) sono riportati i valori di $g$ per ogni inclinazione della slitta, ricavati con la (\ref{eq:10}) 
e i rispettivi errori con la formula generale di propagazione degli errori (\ref{eq:11}).
\begin{equation}
    g=\frac{a}{sin\alpha }
    \label{eq:10}
\end{equation} 
\begin{equation}
    \sigma _{g}=\sqrt{g^2\biggl(\frac{\sigma_a^2}{a^2}+\frac{\sigma_\alpha ^2}{\tan^2\alpha }\biggr)}
    \label{eq:11}
\end{equation}

Nella Figura (\ref{fig:comp}) è riportato $g$ in funzione dell'inclinazione con il relativo errore, la media pesata tra i quattro valori di $g$ e il suo 
errore calcolato con la somma in quadratura tra i $\sigma_{g_i}$.\\
Si nota che i valori calcolati sono incompatibili con il valore dell'accelerazione di gravità noto. 
L'incompatibilità è data, con buona probabilità, da errori sistematici quali una errata calibrazione dell'inclinazione della guidovia ripetuta per tutte le inclinazioni 
o ad errori sistematici legati alla misurazione dei tempi; inoltre anche l'aver trascurato l'attrito con l'aria può aver influito.


\section{Conclusioni}
Si è verificato il moto rettilineo uniformemente accelerato su un piano inclinato e si è stimato il valore dell'accelerazione di 
gravità, purtroppo incompatibile con il valore noto per i motivi descritti sopra.


\newpage
\appendix


\section{Raccolta ed elaborazione dati}\label{sec:tabracdat}

\begin{deluxetable}{cccccc}[ht]
        \tablecaption{Raccolta dati con inclinazione di $15'$}
        
        \tablewidth{0pt}
        \tablehead{
        \multicolumn{1}{p{6em}}{Intervallo tra le fotocellule} & \multicolumn{5}{c}{$\Delta t$}\\
        \colhead{($10^{-2}\,m$)} & \multicolumn{5}{c}{($s$)}
        }
        \startdata
        40.0-50.0  & 0.6530 & 0.6522 & 0.6596 & 0.6579 & 0.6569\\
        40.0-60.0  & 1.2253 & 1.2204 & 1.2079 & 1.2212 & 1.2160\\
        40.0-70.0  & 1.7058 & 1.7201 & 1.7174 & 1.7188 & 1.7306\\
        40.0-80.0  & 2.1619 & 2.1750 & 2.1838 & 2.1762 & 2.1758\\
        40.0-90.0  & 2.5913 & 2.5949 & 2.6040 & 2.6128 & 2.6099\\
        40.0-100.0 & 3.0144 & 3.0076 & 3.0068 & 3.0094 & 3.0166\\
        40.0-110.0 & 3.3815 & 3.3884 & 3.3740 & 3.3989 & 3.3567\\
        40.0-120.0 & 3.7246 & 3.7216 & 3.7316 & 3.7292 & 3.7134\\
        40.0-130.0 & 4.0616 & 4.0701 & 4.0557 & 4.0608 & 4.0626\\
        \enddata
        \tablecomments{L'incertezza sulla misura di intervallo è di $0.1\,cm$. L'incertezza sulla misura di tempo è $10^{-4}\,s$.}
        \label{tab:rac1}
\end{deluxetable}

\begin{deluxetable}{cccccc}[ht]
    \tablecaption{Raccolta dati con inclinazione di $30'$}
    
    \tablewidth{0pt}
    \tablehead{
    \multicolumn{1}{p{6em}}{Intervallo tra le fotocellule} & \multicolumn{5}{c}{$\Delta t$}\\
    \colhead{($10^{-2}\,m$)} & \multicolumn{5}{c}{($s$)}
    }
    \startdata
    40.0-50.0  & 0.4570 & 0.4560 & 0.4576 & 0.4567 & 0.4572\\
    40.0-60.0  & 0.8542 & 0.8527 & 0.8513 & 0.8506 & 0.8506\\
    40.0-70.0  & 1.2026 & 1.2034 & 1.2040 & 1.2038 & 1.2054\\
    40.0-80.0  & 1.5268 & 1.5226 & 1.5207 & 1.5212 & 1.5234\\
    40.0-90.0  & 1.8256 & 1.8246 & 1.8235 & 1.8208 & 1.8186\\
    40.0-100.0 & 2.0954 & 2.0946 & 2.0966 & 2.0940 & 2.1009\\
    40.0-110.0 & 2.3595 & 2.3631 & 2.3615 & 2.3602 & 2.3628\\
    40.0-120.0 & 2.6126 & 2.6060 & 2.6095 & 2.6132 & 2.6144\\
    40.0-130.0 & 2.8452 & 2.8505 & 2.8467 & 2.8475 & 2.8436\\
    \enddata
    \tablecomments{L'incertezza sulla misura di intervallo è di $0.1\,cm$. L'incertezza sulla misura di tempo è $10^{-4}\,s$.}
    \label{tab:rac2}
\end{deluxetable}

\begin{deluxetable}{cccccc}[ht]
    \tablecaption{Raccolta dati con inclinazione di $45'$}
    
    \tablewidth{0pt}
    \tablehead{
    \multicolumn{1}{p{6em}}{Intervallo tra le fotocellule} & \multicolumn{5}{c}{$\Delta t$}\\
    \colhead{($10^{-2}\,m$)} & \multicolumn{5}{c}{($s$)}
    }
    \startdata
    40.0-50.0  & 0.3716 & 0.3718 & 0.3716 & 0.3717 & 0.3712\\
    40.0-60.0  & 0.6926 & 0.6952 & 0.6930 & 0.6928 & 0.6927\\
    40.0-70.0  & 0.9800 & 0.9808 & 0.9780 & 0.9792 & 0.9801\\
    40.0-80.0  & 1.2380 & 1.2395 & 1.2385 & 1.2388 & 1.2386\\
    40.0-90.0  & 1.4823 & 1.4806 & 1.4824 & 1.4826 & 1.4822\\
    40.0-100.0 & 1.7073 & 1.7094 & 1.7060 & 1.7096 & 1.7074\\
    40.0-110.0 & 1.9216 & 1.9223 & 1.9194 & 1.9194 & 1.9201\\
    40.0-120.0 & 2.1236 & 2.1228 & 2.1277 & 2.1254 & 2.1214\\
    40.0-130.0 & 2.3168 & 2.3147 & 2.3140 & 2.3166 & 2.3130\\
    \enddata
    \tablecomments{L'incertezza sulla misura di intervallo è di $0.1\,cm$. L'incertezza sulla misura di tempo è $10^{-4}\,s$.}
    \label{tab:rac3}
\end{deluxetable}

\begin{deluxetable}{cccccc}[ht]
    \tablecaption{Raccolta dati con inclinazione di $45'$ con slitta carica}
    
    \tablewidth{0pt}
    \tablehead{
    \multicolumn{1}{p{6em}}{Intervallo tra le fotocellule} & \multicolumn{5}{c}{$\Delta t$}\\
    \colhead{($10^{-2}\,m$)} & \multicolumn{5}{c}{($s$)}
    }
    \startdata
    40.0-50.0  & 0.3732 & 0.3730 & 0.3730 & 0.3730 & 0.3734\\
    40.0-60.0  & 0.6950 & 0.6952 & 0.6952 & 0.6949 & 0.6952\\
    40.0-70.0  & 0.9824 & 0.9814 & 0.9824 & 0.9816 & 0.9808\\
    40.0-80.0  & 1.2430 & 1.2432 & 1.2423 & 1.2440 & 1.2431\\
    40.0-90.0  & 1.4847 & 1.4856 & 1.4844 & 1.4866 & 1.4850\\
    40.0-100.0 & 1.7126 & 1.7116 & 1.7106 & 1.7107 & 1.7116\\
    40.0-110.0 & 1.9218 & 1.9220 & 1.9236 & 1.9222 & 1.9230\\
    40.0-120.0 & 2.1234 & 2.1242 & 2.1225 & 2.1234 & 2.1240\\
    40.0-130.0 & 2.3158 & 2.3150 & 2.3150 & 2.3160 & 2.3154\\
    \enddata
    \tablecomments{L'incertezza sulla misura di intervallo è di $0.1\,cm$. L'incertezza sulla misura di tempo è $10^{-4}\,s$.}
    \label{tab:rac4}
\end{deluxetable}

\begin{deluxetable}{ccccccccc}[ht]
    \tablecaption{Elaborazione dati}
    \label{tab:racdat}
    \tablewidth{0pt}
    \tablehead{
        \colhead{} & \multicolumn{4}{c}{Inclinazione $15'$} & \multicolumn{4}{c}{Inclinazione $30'$}\\
        \colhead{$\Delta x$} & \colhead{$\bar{t}$} & \colhead{$\Delta t$} & \colhead{$t_{int}$} & \colhead{$v$} & \colhead{$\bar{t}$} & \colhead{$\Delta t$} & \colhead{$t_{int}$} & \colhead{$v$}\\
        \colhead{($10^{-2}\,m$)} & \colhead{($10^{-2}\,s$)} & \colhead{($10^{-2}\,s$)} & \colhead{($10^{-2}\,s$)} & \colhead{($10^{-2}\,m/s$)} & \colhead{($10^{-2}\,s$)} & \colhead({$10^{-2}\,s$)} & \colhead{($10^{-2}\,s$)} & \colhead{($10^{-2}\,m/s$)}
    }
    \startdata
    10.00$\pm$0.05 & 65.59$\pm$0.32  & 65.59$\pm$0.32 & 32.79$\pm$0.15  & 15.24$\pm$0.07 & 45.69$\pm$0.06  & 45.69$\pm$0.06 & 22.84$\pm$0.03  & 21.88$\pm$0.03\\
    20.00$\pm$0.05 & 121.81$\pm$0.66 & 56.22$\pm$0.73 & 93.70$\pm$0.36  & 17.78$\pm$0.23 & 85.18$\pm$0.15  & 39.49$\pm$0.17 & 65.44$\pm$0.08  & 25.31$\pm$0.11\\
    30.00$\pm$0.05 & 171.85$\pm$0.88 & 50.00$\pm$1.00 & 146.83$\pm$0.55 & 19.98$\pm$0.44 & 120.36$\pm$0.13 & 35.17$\pm$0.21 & 102.77$\pm$0.10 & 28.42$\pm$0.17\\
    40.00$\pm$0.05 & 217.45$\pm$0.79 & 45.00$\pm$1.00 & 194.65$\pm$0.59 & 21.92$\pm$0.57 & 152.29$\pm$0.24 & 31.93$\pm$0.28 & 136.32$\pm$0.14 & 31.31$\pm$0.27\\
    50.00$\pm$0.05 & 260.25$\pm$0.93 & 42.00$\pm$1.00 & 238.85$\pm$0.61 & 23.36$\pm$0.66 & 182.26$\pm$0.29 & 29.96$\pm$0.38 & 167.27$\pm$0.19 & 33.36$\pm$0.42\\
    60.00$\pm$0.05 & 301.09$\pm$0.43 & 41.00$\pm$1.00 & 280.67$\pm$0.51 & 24.48$\pm$0.61 & 209.65$\pm$0.27 & 27.38$\pm$0.39 & 195.95$\pm$0.19 & 36.51$\pm$0.53\\
    70.00$\pm$0.05 & 338.00$\pm$1.00 & 37.00$\pm$1.00 & 319.54$\pm$0.82 & 27.00$\pm$1.00 & 236.14$\pm$0.16 & 26.49$\pm$0.31 & 222.89$\pm$0.16 & 37.74$\pm$0.45\\
    80.00$\pm$0.05 & 372.60$\pm$0.86 & 35.00$\pm$1.00 & 355.29$\pm$0.90 & 29.00$\pm$1.00 & 261.11$\pm$0.34 & 24.97$\pm$0.37 & 248.62$\pm$0.18 & 40.04$\pm$0.60\\
    90.00$\pm$0.05 & 406.21$\pm$0.52 & 33.00$\pm$1.00 & 389.41$\pm$0.50 & 29.75$\pm$0.88 & 284.67$\pm$0.26 & 23.55$\pm$0.43 & 272.89$\pm$0.21 & 42.45$\pm$0.77\\
    \enddata
\end{deluxetable}

\begin{deluxetable}{ccccccccc}[ht]
    
    \tablewidth{0pt}
    \tablehead{
        \multicolumn{4}{c}{Inclinazione $45'$} & \multicolumn{4}{c}{Inclinazione $45'$ con slitta carica}\\
        \colhead{$\bar{t}$} & \colhead{$\Delta t$} & \colhead{$t_{int}$} & \colhead{$v$} & \colhead{$\bar{t}$} & \colhead{$\Delta t$} & \colhead{$t_{int}$} & \colhead{$v$}\\
        \colhead{($10^{-2}\,s$)} & \colhead{($10^{-2}\,s$)} & \colhead{($10^{-2}\,s$)} & \colhead{($10^{-2}\,m/s$)} & \colhead{($10^{-2}\,s$)} & \colhead{($10^{-2}\,s$)} & \colhead{($10^{-2}\,s$)} & \colhead{($10^{-2}\,m/s$)}
    }
    \startdata
    37.16$\pm$0.02  & 37.16$\pm$0.02 & 18.57$\pm$0.01  & 26.91$\pm$0.02 & 37.31$\pm$0.02  & 37.31$\pm$0.02 & 18.65$\pm$0.01  & 26.80$\pm$0.01 \\
    69.32$\pm$0.10  & 32.16$\pm$0.11 & 53.24$\pm$0.06  & 31.08$\pm$0.11 & 69.51$\pm$0.01  & 32.19$\pm$0.02 & 53.41$\pm$0.01  & 31.06$\pm$0.02 \\
    97.96$\pm$0.11  & 28.63$\pm$0.15 & 83.64$\pm$0.07  & 34.92$\pm$0.19 & 98.17$\pm$0.07  & 28.66$\pm$0.07 & 83.84$\pm$0.04  & 34.89$\pm$0.09 \\
    123.86$\pm$0.05 & 25.90$\pm$0.12 & 110.91$\pm$0.06 & 38.60$\pm$0.18 & 124.31$\pm$0.06 & 26.14$\pm$0.09 & 111.24$\pm$0.05 & 38.25$\pm$0.13 \\
    148.20$\pm$0.08 & 24.33$\pm$0.10 & 136.03$\pm$0.50 & 41.09$\pm$0.16 & 148.52$\pm$0.09 & 24.21$\pm$0.11 & 136.41$\pm$0.05 & 41.29$\pm$0.18 \\
    170.79$\pm$0.15 & 22.59$\pm$0.17 & 159.49$\pm$0.08 & 44.26$\pm$0.34 & 171.13$\pm$0.08 & 22.60$\pm$0.12 & 159.82$\pm$0.06 & 44.23$\pm$0.23 \\
    192.05$\pm$0.13 & 21.26$\pm$0.20 & 181.42$\pm$0.10 & 47.03$\pm$0.45 & 192.25$\pm$0.08 & 21.11$\pm$0.11 & 181.69$\pm$0.05 & 47.35$\pm$0.25 \\
    212.41$\pm$0.24 & 20.36$\pm$0.28 & 202.23$\pm$0.14 & 49.11$\pm$0.67 & 212.35$\pm$0.07 & 20.09$\pm$0.10 & 202.30$\pm$0.05 & 49.75$\pm$0.25 \\
    231.50$\pm$0.16 & 19.08$\pm$0.29 & 221.96$\pm$0.15 & 52.40$\pm$0.81 & 231.54$\pm$0.05 & 19.19$\pm$0.08 & 221.94$\pm$0.04 & 52.10$\pm$0.22 \\
    \enddata
    \tablecomments{Sono riportati i dati elaborati come spiegato nella sezione Raccolta dati (\ref{sec:dat}).}
    \label{tab:culo}
\end{deluxetable}

\begin{deluxetable}{cc}
    \tablecaption{Compatibilità di $g$}
    \tablewidth{0pt}
    \tablehead{
    \colhead{} & \colhead{$g$}\\
    \colhead{} & \colhead {($m/s^2$)}
    }
    \startdata
    $15'$        & 9.350±0.478\\
    $30'$        & 9.311±0.357\\
    $45'$        & 9.435±0.388\\
    $45'$+massa  & 9.557±0.190\\
    Media pesata & 9.393±0.247\\
    Valore noto  & 9.806
    \enddata
    \tablecomments{Sono riportati i valori di $g$ calcolati per ogni inclinazione. Sono fornite la media pesata dei valori calcolati e il valore noto per un confronto immediato.}
    \label{tab:comp}
\end{deluxetable}


\begin{figure}[ht]
    \centering
    \includegraphics[width=0.75\textwidth]{graf.png}
    \caption{Relazione tra i punti di coordinate $(t_{12}, v(x_1,x_2))$. L'incertezza sulla velocità è predominante rispetto all'incertezza sul tempo, 
    infatti l'incertezza sull'ascissa è delle stesse dimensioni del punto sul grafico, quindi non è apprezzabile.}
    \label{fig:graf}
\end{figure}

\begin{figure}[ht]
    \centering
    \includegraphics[width=0.75\textwidth]{comp.png}
    \caption{Verifica della compatibilità del valore $g$ calcolato sperimentalmente con il valore noto. I due valori non sono compatibili a causa di errori
    sistematici.}
    \label{fig:comp}
\end{figure}


\end{document}
