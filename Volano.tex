\documentclass{aastex63}
\usepackage[T1]{fontenc}
\usepackage{amsmath}
\usepackage[UTF8]{inputenc}
\usepackage{graphicx}
\usepackage{longtable}
\usepackage{wrapfig}
\usepackage{latexsym}
\usepackage{amssymb}
\usepackage{epsf}
\usepackage{hyperref}
%\usepackage{deluxetable}
\usepackage[italian,english]{varioref}
\usepackage[greek,italian]{babel}





\shorttitle{Moto accelerato di un volano}
\shortauthors{Moto accelerato di un volano}

\begin{document}
\title{Moto accelerato di un volano}

\author{Gabriele Bertinelli}
\collaboration{1}{1219907}  

\author{Roben Bhatti}
\collaboration{1}{1216914}

\author{Alberto Brusegan}
\collaboration{1}{1230215}

\email{gabriele.bertinelli@studenti.unipd.it\\ roben.bhatti@studenti.unipd.it\\ alberto.brusegan@studenti.unipd.it}
\affiliation{Dipartimento Fisica e Astronomia, UniPD\\
Corso di Astronomia, a.a. 2019-2020}

\section{Introduzione}
Il volano è un corpo rigido vincolato a ruotare attorno ad un asse fisso, cui è collegato da un sistema di cuscinetti a sfere; 
un peso in ottone, al quale è fissato un filo di refe che si può avvolgere attorno ad un cilindro munito di una scanalatura elicoidale e 
solidale al volano stesso, viene poi sfruttato per applicare una forza di momento noto rispetto all'asse di rotazione.

\section{Obiettivi}
\begin{enumerate}
    \item Verifica della legge del moto;
    \item misura del momento d'inerzia del volano rispetto all'asse di rotazione;
    \item misura del momento delle forze d'attrito (sempre rispetto all'asse fisso).
\end{enumerate}

\section{Dettagli tecnici sull'apparato sperimentale}
Uno schema dell'apparato strumentale è riportato in Fig.\ref{fig:volano}.

\begin{figure}[h!]
    \centering
    \includegraphics[width=0.30\textwidth]{volano.jpg}
    \caption{Schema di un volano}
    \label{fig:volano}
\end{figure}

Le forze agenti sul volano sono il momento delle forze d'attrito, la tensione del filo $T$, la forza peso $P$ e la tensione $T'$.
Trattandosi di attrito tra corpi solidi (rotolamento delle sfere dei cuscinetti), il momento delle forze d'attrito che agiscono 
sull'asse del volano si può ritenere indipendente dalle condizioni del moto: quindi $M_a$ è una costante.\\
Valori noti:\\
$m=34.0\pm0.5\,\,g$ è la massa del peso;\\ 
$r=18.95\pm0.01\,\,mm$ (si veda Fig.\ref{fig:volano} per il significato geometrico di $r$);\\
$g=9.806\pm0.001\,\,m/s^2$ è il valore della accelerazione di gravità attesa a Padova.\\
Le equazioni del moto sono pertanto:
\begin{equation}
\begin{array}{ccc}
    m\,g-T'=m\,a \\ 
    T\,r-M_a=I\frac{d\omega}{dt}
\end{array}
\label{eq:1.1}
\end{equation}

$\frac{d\omega}{dt}$ lo possiamo sostituire con $\frac{a}{r}$ e il filo che connette il peso al volano si può considerare di massa trascurabile,
risulterà in conseguenza $T=T'$. Dopo aver fatto le dovute sostituzioni ed esplicitando per $a$, accelerazione del peso, troviamo 
\begin{equation}
    a=\frac{(m\,r\,g-M_a)r}{I+m\,r^2}
    \label{eq:1.2}
\end{equation}
dove $I$ è il momento di inerzia del volano.\\
Dal momento che tutte le grandezza da cui $a$ dipende sono costanti il moto del volano sarà uniformemente accelerato. Se il filo si considera 
inestensibile, lo spazio percorso dal peso è legato all'angolo $\phi$ di cui è ruotato il volano
\begin{equation}
    s=\frac{1}{2}a\,t^2=r\,\phi
    \label{eq:1.3}
\end{equation}
La velocità angolare $\omega$ del volano in funzione del tempo è espressa da 
\begin{equation}
    \omega(t)=\frac{a}{r}t=\frac{m\,r\,g-M_a}{I+m\,r^2}t=\alpha\,t
    \label{eq:1.4}
\end{equation}
in cui 
\begin{equation}
    \alpha=\frac{m\,r\,g-M_a}{I+m\,r^2}
    \label{eq:1.5}
\end{equation}
è l'accelerazione angolare del moto.\\
Esprimendo l'angolo di rotazione in funzione dei giri del volano, $1/2\,a\,t^2=2\,\pi\,r\,n$, si ricava 
\begin{equation}
    \theta=2\sqrt{\frac{\pi(I+m\,r^2)}{m\,r\,g-M_a}}
    \label{eq:1.6}
\end{equation}
Il momento di inerzia $I$ si ricava dalla (\ref{eq:1.6}); bisogna però conoscere $M_a$, esso è legato, in ogni istante, al valore dell'accelerazione angolare (negativa, perché misurano in decelerazione) dalla
\begin{equation}
    -M_a=I\frac{d\omega}{dt}
\end{equation}
Integrando
\begin{equation}
    \omega(t)=\omega_{max}-\frac{M_a}{I}t=\omega_{max}+\beta\,t
    \label{eq:1.8}
\end{equation}
dove 
\begin{equation}
    \beta=-\frac{M_a}{I}
    \label{eq:1.9}
\end{equation}
rappresenta la decelerazione del moto.

\section{Raccolta dati}
La raccolta dati è stata effettuata come quanto segue:
\begin{enumerate}
    \item Avvolgere il filo di refe attorno al cilindro solidale al volano e lasciare quest'ultimo libero di ruotare nello stesso istante in cui si avvia il cronometro.
    \item Acquisire i tempi in corrispondenza della fine di ognuno dei giri compiuti dal volano sino al 13° giro.
    \item Attendere il distacco del peso dal volano.
    \item Acquisire i tempi di decelerazione per effettuare 10,20,...,60 giri a partire dal distacco del peso.
\end{enumerate}
Ripetere i punti 1-4 per dodici volte.\\
Le misure dei tempi sono state prese con un cronometro digitale con incertezza $1\cdot 10^{-4}\,s$.\\
I dati raccolti, essendo uguali per tutti i gruppi, non vengono riportati.

    
\section{Elaborazione dati}\label{sec:elab}
I dati sono stati elaborati nel seguente modo:
\begin{enumerate}
    \item Calcolare i tempi medi di percorrenza di 1, 2,.., 13 giri nella fase di accelerazione rispetto le dieci
    misure ripetute e i tempi medi di percorrenza di 10, 20, ..., 60 giri nella fase di decelerazione.
    \item Calcolare la differenza $\Delta t_i$ corrispondente al tempo necessario per effettuare il giro i-esimo (i=1, ..., 13) nella fase di accelerazione. Identicamente
          si è calcolato $\Delta t_{d_i}$ (i=10,20,...,30), tempo necessario per compiere 10 giri nella fase di decelerazione.
    \item Calcolare la velocità media angolare $\omega_i$ per effettuare il giro i-esimo: $\omega_i=2\pi/\Delta t_i$ nella fase di accelerazione. $\omega_i$ corrisponde alla velocità istantanea del volano relativa all'istante intermedio $t_{i,ist}$ dell'intervallo di tempo considerato in corrispondenza di ogni giro.
    \item \label{itm:a} Nel grafico, riportato nella Fig.(\ref{fig:fit}), si sono disposte le coppie ($t_{i,ist};\,\omega_i$) e si è interpolato con una retta di equazione $\omega(t)=\omega_0+\alpha\,t$; la pendenza $\alpha$ è stata calcolata in (\ref{eq:1.5}).
    \item Analogamente si è fatto per la fase di decelerazione: la velocità media angolare $\omega_{i_d}$ per effettuare 10 giri nella fase di decelerazione vale $\omega_{i_d}=(2\cdot 10\,pi)/t$.
    $\omega_{i_d}$ corrisponde alla velocità istantanea del volano relativa all'istante intermedio $t_{i_d,ist}$ dell'intervallo di tempo considerato ogni 10 giri. 
    \item Nel grafico della Fig.(\ref{fig:fit}) si sono disposte anche le coppie ($t_{i,ist};\,\omega_i$), della fase di decelerazione ed interpolate con la retta di equazione $\omega(t)=\omega_{max}+\beta\,t$; la pendenza $\beta$ è stata calcolata in (\ref{eq:1.9}).
    \item Ricavare $I$ utilizzando i valori di $\alpha$ e $\beta$
          \begin{equation}
              I=\frac{m\,r\,g-\alpha\,m\,r^2}{\alpha-\beta}
              \label{eq:1.10}
          \end{equation}
    associando ad $I$ l'errore dato dalla formula di propagazione dell'errore.
    \begin{equation}
    \begin{split}
        \sigma_I^2&=\biggl(\frac{r\,g-\alpha\,r^2}{\alpha-\beta}\sigma_m\biggr)^2+\biggl(\frac{m\,g-2\,\alpha\,m\,r}{\alpha-\beta}\sigma_r\biggr)^2+\\
                    &+\biggl(\frac{m\,r}{\alpha-\beta}\sigma_g\biggr)^2+\biggl(\frac{m\,r^2\,(\alpha-\beta)+m\,g\,r-\alpha\,m\,r^2}{(\alpha-\beta)^2}\sigma_{\alpha}\biggr)^2+\\
                    &+\biggl(\frac{m\,r\,g-\alpha\,m\,r^2}{(\alpha-\beta)^2}\sigma_{\beta}\biggr)^2
    \end{split}
    \end{equation}
    Dove gli errori su $\alpha$ e $\beta$ sono 
    \begin{equation}
        \sigma_{\alpha}=\sigma_{\omega}\sqrt{\frac{1}{\sum t_i^2- (\sum t_i)^2}}
    \end{equation}
    \begin{equation}
        \sigma_{\beta}=\sigma_{\omega}\sqrt{\frac{1}{\sum t_i^2- (\sum t_i)^2}}
    \end{equation}
    \item Ricavare $M_a$ da $\beta$ con relativo errore, sempre dato dalla formula di propagazione dell'errore. 
    \begin{equation}
        \sigma_{M_a}=\sqrt{(-\beta\,\sigma_I)^2+(-I\,\sigma_{\beta})^2}
    \end{equation}
\end{enumerate}

Vediamo più in dettaglio i passaggi svolti per ricavare i dati che sono riportati nella Tabella \ref{tab:elabdat1} e \ref{tab:elabdat2}.\\
Si è calcolato $\bar{t}$, tempo medio, con (\ref{eq:2.1}) ovvero la media aritmetica tra le 12 misure effettuate per ogni prova, relativo ad ogni giro.\\
L'errore associato è stato ricavato tramite la (\ref{eq:2.2}), cioè la deviazione standard corretta (al denominatore $N-1$ invece di $N$). Si è scelto di utilizzare 
questa correzione per ovviare la tendenza della formula "classica" a sottostimare le incertezze, soprattutto nel caso in cui si lavori con pochi dati, come quello che stiamo analizzando.
\begin{equation}
    \bar{t}=\frac{\sum_{n=1}^{12}t_i}{N}
    \label{eq:2.1}
\end{equation}
\begin{equation}
    \sigma_{\bar{t}}=\sqrt{\frac{\sum_{n=1}^{N}(t_i-\bar{t})^2}{N-1}}
    \label{eq:2.2}
\end{equation}

Si è poi calcolato $\Delta t$ con la (\ref{eq:2.3}) ovvero la differenza tra $\bar{t}$ relativo a quel giro e il precedente. 
Come errore su $\Delta t$ si è usata la (\ref{eq:2.4}), somma in quadratura.
\begin{equation}
    \Delta t={\bar{t}_i-\bar{t}_{i-1}}
    \label{eq:2.3}
\end{equation}
\begin{equation}
    \sigma_{\Delta t}=\sqrt{\sigma _{\bar{t}_i}^2+\sigma _{\bar{t}_{i-1}}^2}
    \label{eq:2.4}
\end{equation}

Si è calcolato il $t_{int}$, tempo intermedio, ovvero l'istante in cui la velocità istantanea è uguale alla velocità angolare calcolata nel $\Delta t$. 
Esso viene calcolato con la (\ref{eq:2.5}), cioè con media aritmetica tra $\bar{t}_i$ e $\bar{t}_{i-1}$ e l'errore associato si calcola 
con la (\ref{eq:2.6}) ovvero con la somma in quadratura moltiplicata per un fattore di $\frac{1}{2}$.
\begin{equation}
      t_{int}=\frac{\bar{t}_i+\bar{t}_{i-1}}{2}
      \label{eq:2.5}
\end{equation}
\begin{equation}
    \sigma _{t_{int}}=\frac{1}{2}\sqrt{\sigma_{\bar{t}_i}^2+\sigma _{\bar{t}_{i-1}}^2}
    \label{eq:2.6}
\end{equation}

In seguito si è calcolata la velocità angolare con la (\ref{eq:2.7}), rapportando $s\,\pi$ a $\Delta t$.
L'errore associato si ricava tramite la (\ref{eq:2.8}), formula generale della propagazione degli errori.
\begin{equation}
    \omega=\frac{2\,pi}{\Delta t_{i}}
    \label{eq:2.7}
\end{equation}
\begin{equation}
    \sigma _{\omega}=\sqrt{\biggl(-\frac{2\,\pi}{\Delta t^2}\sigma_{\Delta t}\biggr)^2}
    \label{eq:2.8}
\end{equation}

Gli stessi passaggi sono stati svolti per la fase di decelerazione, tenendo conto che $\omega_d=(2\cdot 10\,\pi)/\Delta t$ e quindi l'errore associato è $\sigma _{\omega_d}=\sqrt{\biggl(-\frac{2\cdot 10\,\pi}{\Delta t^2}\sigma_{\Delta t}\biggr)^2}$.\\
I successivi passaggi sono stati descritti a partire dal punto (\ref{itm:a}).\\
I valori di $\alpha$, $\beta$, $I$ e $M_a$, e relative incertezze, sono riportate nella Tabella \ref{tab:fin}.\\
Confrontando il valore di $I$ noto e quello ricavato \ref{eq:1.10}, possiamo concludere che i due valori sono compatibili, come è anche possibile notare dalla Fig.\ref{fig:comp}.


\section{Conclusioni}
Si è verificata con successo la compatibilità del momento d'inerzia del volano rispetto all'asse di rotazione. 
Si è ricavato il momento d'attrito rispetto all'asse di rotazione con il suo errore e infine si è verificata la 
legge del moto circolare uniformemente accelerato.


\newpage
\appendix


\section{Elaborazione dati}\label{sec:elabdat}

\begin{table}[h!]
    \centering
    \caption{Elaborazione dati in fase di accelerazione}
      \begin{tabular}{cccc}
      \toprule
      \multicolumn{1}{c}{$\bar{t}$} & \multicolumn{1}{c}{$\Delta t$} & \multicolumn{1}{c}{$t_{ist}$} & $\omega$ \\
      \multicolumn{1}{c}{($s$)} & \multicolumn{1}{c}{($s$)} & \multicolumn{1}{c}{($s$)} & ($rad/s$) \\
      \hline
      16.8428$\pm$0.3224 & 16.8428$\pm$0.3224 & 16.8428$\pm$0.1612 & 0.3730$\pm$0.3224 \\
      17.8428$\pm$0.3676 & 7.0196$\pm$0.4889 & 20.3526$\pm$0.2445 & 0.8951$\pm$0.4800 \\
      18.8428$\pm$0.3037 & 5.5267$\pm$0.4768 & 26.6259$\pm$0.2383 & 1.1369$\pm$0.4768 \\
      19.8428$\pm$0.2854 & 4.5939$\pm$0.4167 & 31.6862$\pm$0.2084 & 1.3677$\pm$0.4167 \\
      20.8428$\pm$0.3091 & 4.0088$\pm$0.4207 & 35.9876$\pm$0.2104 & 1.5673$\pm$0.4207 \\
      21.8428$\pm$0.3252 & 3.6525$\pm$0.4487 & 39.8182$\pm$0.2243 & 1.7203$\pm$0.4487 \\
      22.8428$\pm$0.3249 & 3.3486$\pm$0.4597 & 43.3188$\pm$0.2298 & 1.8764$\pm$0.4597 \\
      23.8428$\pm$0.3423 & 3.1138$\pm$0.4719 & 46.5500$\pm$0.2360 & 2.0179$\pm$0.4719 \\
      24.8428$\pm$0.4170 & 2.8602$\pm$0.5395 & 49.5370$\pm$0.2698 & 2.1968$\pm$0.5395 \\
      25.8428$\pm$0.3534 & 2.8391$\pm$0.5466 & 52.3866$\pm$0.2733 & 2.2131$\pm$0.5466 \\
      26.8428$\pm$0.3626 & 2.6057$\pm$0.5087 & 55.1090$\pm$0.2532 & 2.4114$\pm$0.5064 \\
      27.8428$\pm$0.3568 & 2.5301$\pm$0.5087 & 57.6769$\pm$0.2544 & 2.4833$\pm$0.5087 \\
      28.8428$\pm$0.3442 & 2.3529$\pm$0.4958 & 60.1184$\pm$0.2479 & 2.6704$\pm$0.4958 \\
      \hline
      
      \end{tabular}
    \tablecomments{Sono riportati i dati elaborati come spiegato nella Sezione \ref{sec:elab}.}
    \label{tab:elabdat1}
  \end{table}


  \begin{table}[h!]
    \centering
        \caption{Elaborazione dati in fase di decelerazione}
        \begin{tabular}{cccc}
            \toprule
            \multicolumn{1}{c}{$\bar{t}$} & \multicolumn{1}{c}{$\Delta t$} & \multicolumn{1}{c}{$t_{ist}$} & $\omega$ \\
            \multicolumn{1}{c}{($s$)} & \multicolumn{1}{c}{($s$)} & \multicolumn{1}{c}{($s$)} & ($rad/s$) \\
            \hline
            24.0383$\pm$0.7564 & 24.0383$\pm$0.7564 & 24.0383$\pm$0.3781 & 2.6138$\pm$0.1645 \\
            48.6321$\pm$0.7188 & 24.5937$\pm$1.0434 & 36.3352$\pm$0.5217 & 2.5548$\pm$0.2168 \\
            73.9499$\pm$0.7455 & 25.3178$\pm$1.0355 & 61.2910$\pm$0.5178 & 2.4817$\pm$0.2030 \\
            99.9417$\pm$0.7581 & 25.9919$\pm$1.0632 & 86.9458$\pm$0.5316 & 2.4174$\pm$0.1978 \\
            126.6681$\pm$0.8106 & 26.7264$\pm$1.1099 & 113.3049$\pm$0.5549 & 2.3509$\pm$0.1953 \\
            154.1599$\pm$0.8642 & 27.4918$\pm$1.1849 & 140.4140$\pm$0.5925 & 2.2855$\pm$0.1970 \\
            \hline
            \end{tabular}%
    \tablecomments{Sono riportati i dati elaborati come spiegato nella Sezione \ref{sec:elab}.}
    \label{tab:elabdat2}%
\end{table}%
  

\begin{table}[h!]
    \centering
    \caption{Valori delle incognite}
      \begin{tabular}{cccc}
      \toprule
      $\alpha$ & $\beta$  & $I$ & \multicolumn{1}{c}{$M_a$} \\
      $rad/s^2$ & $rad/s^2$ & ($kg\,m^2$) & \multicolumn{1}{c}{$N\,m$} \\
      \hline
      0.0474$\pm$0.0072 & -0.0027$\pm$0.0001 & 0.1260$\pm$0.0180 & 0.0003$\pm$0.0001 \\
      \hline
      \end{tabular}%
      \tablecomments{Vengono riportati i valori di $\alpha$, $\beta$, $I$ e $M_a$ con i rispettivi errori. \\Si verifica che c'è compatibilità tra 
      $I$ ricavato e $I$ noto ($0.1392\pm 0.0001$), \\come si può notare graficamente dalla Fig.\ref{fig:comp}.}
    \label{tab:fin}%
  \end{table}%
  
  



  \begin{figure}[h!]
    \centering
    \includegraphics[width=0.95\textwidth]{Fit.png}
    \caption{Grafico in cui sono rappresentate le coppie ($t_{i,ist};\,\omega_i$), gli errori relativi a $\omega_i$, sia in fase di accelerazione che decelerazione e le rispettive rette interpolanti.}
    \label{fig:fit}
\end{figure}

\begin{figure}[h!]
    \centering
    \includegraphics[width=0.70\textwidth]{comp.jpeg}
    \caption{Compatibilità della misura di $I$, con il valore noto. Si può notare che $I$ noto con la sua barra d'errore sta all'interno della zona delimitata dall'incertezza di $I$ ricavato, indicando la buona riuscita dell'esperimento e quindi la compatibilità delle due misure.}
    \label{fig:comp}
\end{figure}

\end{document}
