\documentclass{aastex63}
\usepackage[T1]{fontenc}
\usepackage{amsmath}
%\usepackage[UTF8]{inputenc}
\usepackage{graphicx}
\usepackage{longtable}
\usepackage{wrapfig}
\usepackage{latexsym}
\usepackage{amssymb}
\usepackage{epsf}
\usepackage{hyperref}
\usepackage{physics}
%\usepackage{deluxetable}
\usepackage[italian,english]{varioref}
\usepackage[greek,italian]{babel}


\shorttitle{Focale lente convergente}
\shortauthors{Gabriele Bertinelli}

\begin{document}
\title{Misura della focale di una lente convergente sottile}

\author{Gabriele Bertinelli - 1219907}
\collaboration{1}{}  

\email{gabriele.bertinelli@studenti.unipd.it}
\affiliation{Dipartimento Fisica e Astronomia, UniPD\\
Corso di Astronomia, a.a. 2020-2021}

\section{Introduzione}
Una lente è un elemento ottico, generalmente in vetro o materiali plastici, che ha la proprietà di concentrare o di far divergere i raggi di luce.
In particolare, questa relazione è incentrata sulla misura della focale di una lente convergente biconvessa, ovvero con le proprietà di far convergere
dei raggi luminosi collimati (paralleli all'asse) in un punto dell'asse, oltre la lente, chiamato \textit{punto focale}; la distanza tra la lente sottile e il fuoco si chiama \textit{lunghezza focale}.\\
La caratteristica di \textit{lente sottile} è utile nell'analisi dei dati perché permette di non considerare ciò che accade all'interno della lente.\\
Generalmente si possono utilizzare per la misura diversi metodi:
\begin{enumerate}
    \item metodo della collimazione;
    \item metodo dei punti coniugati;
    \item metodo di Bessel;
    \item variante di Silbermann.
\end{enumerate}
Verranno trattati solo i primi due metodi.

\section{Raccolta dati}
In questa sezione vengono analizzati i due metodi di misura.
\subsection{Metodo della collimazione - Accenni teorici}
Si consideri una lente convergente e si ponga una sorgente di luce a sinistra della lente, tanto lontana da portela considerare all'infinito e i suoi raggi paralleli.
Si sa che tale lente concentrerà il fascio collimato in un punto detto fuoco secondario ($F_2$), a destra della lente. Ma la lente è dotata di due fuochi, per cui se si pone la sorgente nel fuoco primario ($F_1$)
essa genera un fascio di raggi che viene collimato dalla lente in un fascio di raggi paralleli all'asse ottico, che emerge a destra della lente.\\
Per una lente simmetrica, come nel caso preso in esame, le distanze focali coincidono e nel limite di spessore nullo della lente (i.e. sul bordo esterno) sono uguali alla distanza focale della lente, che è quella per cui vale l'espressione dei \textit{punti coniugati}.\\
Dunque, se si pone la lente a una distanza dalla sorgente puntiforme tale che il fascio emergente sia collimato, quella distanza è proprio la lunghezza focale della lente.

\subsection{Metodo dei punti coniugati - Accenni teorici}\label{subsec:coniugatiteor}
Questo metodo sfrutta l'equazione dei punti coniugati:
\begin{equation*}
    \frac{1}{p}+\frac{1}{q}=\frac{1}{f}
\end{equation*}
In cui $p$ è la distanza tra la sorgente puntiforme e la lente; $q$ è la distanza tra la lente e lo schermo; $f$ è la distanza focale. (Fig. \ref{fig:schemabanco})\\
Riscrivendo l'equazione, esplicitando $f$, si ottiene
\begin{equation*}
    f=\frac{pq}{p+q}
\end{equation*}

\subsection{Metodo della collimazione - Accenni pratici}\label{subsec:collim}
Si provvede ad avvitare la lente nell'apposito supporto e a centrarla in modo che il suo centro cada in corrispondenza della tacca bianca del supporto. 
Per il centraggio si regola la vite micrometrica del supporto in posizione di azzeramento indicata sul supporto ($\Delta_{micrometro}$), cui si aggiunge il semispessore minimo della lente $\frac{dr}{2}$.\\
Dopo aver posizionato la mascherina puntiforme a ridosso della sorgente, si pone la lente a diverse distanze di $p$ dalla sorgente, fino a trovare la distanza alla quale il fascio è collimato. Per controllare la collimazione
del fascio si fa scorrere lo schermo lungo la rotaia, controllando che l'immagine che si forma su di esso mantenga sempre lo stesso diametro. È possibile determinare il massimo e il minimo valore corretto di $p$, per ottenere una stima di $f$.\\
Questo metodo è poco accurato e, come si vedrà nella sezione di analisi dati, si discosterà significativamente dal valore reale.

\subsection{Metodo dei punti coniugati - Accenni pratici}\label{subsec:coniugatiprat}
Dopo aver posizionato la mascherina puntiforme a ridosso della sorgente e il diaframma (Fig. \ref{fig:schemabanco}), per mettersi nella condizione di raggi parassiali, si provvede a posizionare la lente a una distanza $p_k$, con $k=1,2,...,10$, dalla sorgente. 
Si porta lo schermo vicino alla lente e si comincia ad allontanarlo finché i due punti-immagine diventano uno solo. Si misura la posizione dello schermo, indicando questo valore come $q_{k,min}$. A causa della profondità di campo, esisterà un $q_{k,max}$ oltre il quale il punto si sfuoca di nuovo in due punti. 
Si calcola la media tra $q_{k,min}$ e $q_{k,max}$ e si considera quella come valore di $q_k$ associato a $p_k$. 

\section{Analisi dati}
\subsection{Metodo della collimazione}
Nella sezione \ref{subsec:collim} si è descritto il metodo di raccolta dati. Si sono calcolate 10 coppie di $s_{k,min}$ e $s_{k,max}$, ovvero la distanza della lente dallo $0$ della scala graduata, per ogni posizione dello schermo $h_k$ (con $k=1,...,10$). Le misure sono raccolte nella Tabella ().\\
La misura di $p_{k,min/max}$, ovvero la distanza tra la sorgente e la lente (che per questo metodo sarà anche una stima della focale $f$), è data da $p_{k,min/max}=s_k-d$ (per ogni valore di $s_{k,min/max}$), dove $d$ è la distanza della sorgente dallo $0$ della scala graduata.\\
Il valore di $p_k$ si calcola operando una media aritmetica 
\begin{equation}
    p_k=\frac{p_{k,min}+p_{k,max}}{2}
    \label{eq:pk}
\end{equation}
Per avere una stima della lunghezza focale $f$, ovvero di $p$, si calcola nuovamente la media aritmetica tra i valori di $p_k$, trovati tramite \ref{eq:pk}
\begin{equation}
    p=\frac{\sum_{k=1}^{10} p_k}{10}
\end{equation}
L'errore sulla misura di $p_k$ $\sigma_{p_k}$ è dato dalla somma dell'errore di sensibilità della posizione della sorgente $\sigma_{source}$ e dell'errore di sensibilità sulla posizione della lente $\sigma_{lens}$.
\begin{equation}
    \sigma_{p_k}=\sigma_{source}+\sigma_{lens}
\end{equation}
L'errore sulla misura finale di $p$ (o $f$) è dato dalla deviazione standard
\begin{equation}
    \sigma_p=\sqrt{\frac{\sum_{k=1}^{10}(p-p_k)^2}{9}}
\end{equation}
Nella Sezione \ref{sec:appendix} sono riportate le misure graficate.

\subsection{Metodo dei punti coniugati}
Nella Sezione \ref{subsec:coniugatiteor} veniva scritta l'equazione dei punti coniugati che riportiamo per semplicità
\begin{equation}
    f=\frac{pq}{p+q}
    \label{eq:punticon}
\end{equation}
Calcoliamo ora l'errore massimo associato $\Delta f$ utilizzando la propagazione degli errori:
\begin{equation}
    \Delta f=\abs{\frac{\partial f}{\partial p}}\Delta p + \abs{\frac{\partial f}{\partial q}}\Delta q
\end{equation}
Svolgendo i calcoli diventa
\begin{equation}
    \Delta f= \frac{q^2}{(p+q)^2}\Delta p + \frac{p^2}{(p+q)}\Delta q
\end{equation}
Nella Sezione \ref{subsec:coniugatiteor} è stato discusso il metodo pratico di raccolta dati.\\
Per ogni posizione della lente $p_k$, con $k=1,...,10$, si raccolgono 10 coppie di misure $[\,q_{kj,min},q_{kj,max}]\,$, tali che i due punti luminosi siano sempre sovrapposti. Per ogni coppia $[\,q_{kj,min},q_{kj,max}]\,$
si calcola il valore medio $q_{k,j}$. Ad ogni valore $p_k$ verrà assegnata la misura $q_k$ data dalla media aritmetica dei $q_{k,j}$. Algebricamente 
\begin{subequations}
    \begin{align}
    q_{k,j}&=\frac{q_{kj,min}+q_{kj,max}}{2} \quad \quad j=1,...,10\\
    q_k&=\frac{\sum_{j=1}^{10}q_{k,j}}{10} \\
    \sigma_{q_k}&=\sqrt{\frac{\sum_{j=1}^{10}(q_{k,j}-q_k)^2}{9}} \label{eq:sigmaqk}
    \end{align}
\end{subequations}
La \ref{eq:sigmaqk} è l'errore sulla misura $q_k$ associata al valore $p_k$.\\
A causa dell'effetto combinato di profondità di campo e errore di misura, si è usato l'errore massimo $\Delta q_k=3\sigma_{q_k}$.\\
Per ogni valore di $k$ si è calcolato il corrispondente valore della lunghezza focale $f_k$ e del suo errore $\Delta f_k$. Si è calcolato, infine, la media pesata degli $f_k$ per ricavare la lunghezza focale della lente $f$:
\begin{subequations}
    \begin{align}
        f&=\frac{\sum_{k=1}^{10}\omega_k f_k}{\sum_{k=1}^{10}\omega_k}\\
        \sigma_{f_k}&=\frac{1}{\sqrt{\sum_{k=1}^{10}\omega_k}}\\
        \omega_k&=\bigg(\frac{1}{\sigma_{f_k}}\bigg)^2
    \end{align}
\end{subequations}
Si è tenuto conto del fatto che $\Delta f_k=3\sigma_{f_k}$ e $\Delta f=3\sigma_f$.








\newpage
\appendix
\restartappendixnumbering
\section{Tabelle e figure}\label{sec:appendix}

\begin{figure}[h!]
    \centering
    \includegraphics[width=0.60\textwidth]{schema.jpg}
    \caption{Rappresentazione schematica del banco ottico per il \textit{metodo dei punti coniugati}. Per il \textit{metodo della collimazione} bisogna considerare lo schema a meno del \textit{diaframma D4}.}
    \label{fig:schemabanco}
\end{figure}

\end{document}