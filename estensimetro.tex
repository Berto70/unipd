\documentclass{aastex63}
\usepackage[T1]{fontenc}
\usepackage[UTF8]{inputenc}
\usepackage{graphicx}
\usepackage{longtable}
\usepackage{wrapfig}
\usepackage{latexsym}
\usepackage{amssymb}
\usepackage{epsf}
\usepackage{hyperref}
\usepackage[italian,english]{varioref}
\usepackage[greek,italian]{babel}
\usepackage[suftesi]{frontespizio}




\shorttitle{Verifica della legge di Hooke}
\shortauthors{Verifica della legge di Hooke}

\begin{document}
\title{Verifica della legge di Hooke}

\author{Gabriele Bertinelli}
\collaboration{1}{1219907}  

\author{Roben Bhatti}
\collaboration{1}{1216914}

\author{Alberto Brusegan}
\collaboration{1}{1230215}

\email{gabriele.bertinelli@studenti.unipd.it\\ roben.bhatti@studenti.unipd.it\\ alberto.brusegan@studenti.unipd.it}
\affiliation{Dipartimento Fisica e Astronomia, UniPD\\
Corso di Astronomia, a.a. 2019-2020}

\section{Introduzione}
La legge di Hooke descrive la deformazione di un corpo solido (es. cilindro retto) vincolato quando viene sottoposto all'azione di una forza.
Questa definizione è proporzionale al modulo della forza applicata, almeno fino a che la forza esercitata non superi i limiti di elasticità e di rottura.
Un corpo si dice elastico se la sua deformazione segue una legge lineare, la legge di Hooke che andremo a verificare, in funzione della forza applicata.
Quindi a un cilindro retto, nel nostro caso un filo, di materiale elastico la legge di Hooke impone che si allunghi o accorci di una quantità $x-x_0=K\,F$, dove $x_0$ rappresenta la lunghezza del corpo a riposo e $K$ è la costante elastica.
Per un filo si dimostra che $K$ è inversamente proporzionale alla sezione e al modulo di Young $(E)$: $E=\frac{x_0}{S\,K}$.
\section{Obiettivi}
\begin{enumerate}
\item la verifica della risposta elastica lineare, ovvero la dipendenza lineare del modulo 
dell'allungamento (accorciamento) $\Delta x$ di un corpo elastico dal modulo della forza applicata $F_{app}$: 
    $\Delta x=K\,F_{app}$;
\item nel caso di un corpo cilindrico, stima del modulo di Young $E=\frac{x_0}{S\,K}$, dove $S$ è la sezione del cilindro, $K$ è la costante
    di proporzionalità di $\Delta x$ e $F$ e $x_0$ la lunghezza del filo a riposo.
\end{enumerate}

\section{Dettagli tecnici sull'apparato sperimentale}
L'estensimetro è uno strumento che permette di misurare l'allungamento subito da un filo elastico a seguito
dell'applicazione di una forza di intensità nota ad un estremità dello stesso (l'altra estremità essendo invece
vincolata). Lo schema rappresentativo dell'apparato è riportato in Figura~\ref{fig:schema}.

\begin{figure}
\plotone{schemaestensimetro.pdf}
\caption{Schema di un estensimetro}
\label{fig:schema}
\end{figure}

La misura dell'allungamento del filo è effettuata attraverso un minimetro, comparatore associato ad una
lancetta rotante: 1 giro completo della lancetta corrisponde ad $1\,mm$ di allungamento del filo (sensibilità al
centesimo di millimetro). Allungamenti superiori ad $1\,mm$ (n. giri completi) sono visualizzati nella finestrella
all'interno del quadrante.\\
La forza viene applicata al filo mediante un dinamometro attraverso la rotazione di una ghiera: la forza
effettiva applicata $F_{app}$ è pari a quattro volte il valore $F$ letto sulla scala graduata, che è espressa in grammi-peso.\\
In laboratorio sono disponibili 16 estensimetri in acciaio ($E_{acciaio}=(20.5\pm0.1)\cdot10^{10}\,N/m^2$), due estensimetri in tungsteno ($E_{tung}=(37.9\pm1.5)\cdot10^{10}\,N/m^2$) ed uno di ottone ($E_{ottone}=(9.6\pm0.2)\cdot10^{10}\,N/m^2$)

\begin{deluxetable}{cccc}[h!]
    \tablecaption{Caratteristiche del filo}\label{tab:filo}
    \tabletypesize{\scriptsize}
    \tablewidth{0pt} 
    \tablehead{ 
    \colhead{N°} & \colhead{Materiale} & \colhead{Lunghezza} & \colhead{Diametro}\\
    \colhead{} & \colhead{} & \colhead{$(mm)$} & \colhead{$(mm)$}
    }
    \startdata
    1  & tungsteno & 1000$\pm$2 & 0.250$\pm$1\%\\
    3  & ottone    & 1000$\pm$2 & 0.500$\pm$1\%\\
    6  & acciaio   & 950$\pm$2  & 0.305$\pm$1\%\\
    7  & acciaio   & 950$\pm$2  & 0.330$\pm$1\%\\
    10 & acciaio   & 950$\pm$2  & 0.406$\pm$1\%\\
    12 & acciaio   & 950$\pm$2  & 0.279$\pm$1\%\\
    14 & acciaio   & 800$\pm$2  & 0.279$\pm$1\%\\
    15 & acciaio   & 700$\pm$2  & 0.279$\pm$1\%\\
    17 & acciaio   & 500$\pm$2  & 0.279$\pm$1\%\\
    18 & acciaio   & 400$\pm$2  & 0.279$\pm$1\%\\
    19 & acciaio   & 300$\pm$2  & 0.279$\pm$1\%\\
    \enddata
    \tablecomments{Sono forniti i valori di lunghezza a riposo $x_0$ e diametro $D$ caratteristici di ciascun estensimetro.}
\end{deluxetable}

\[
\]

\section{Raccolta dati}
Nella sezione \ref{sec:tabracdat} è fornita la tabella della raccolta dati degli 11 estensimetri da noi analizzati, le cui caratteristiche sono riportate nella tabella \ref{tab:filo}.\\
\'E bene precisare che solo i dati riguardanti l'estensimetro $n$° $15$ sono stati raccolti dal nostro gruppo; gli altri dati sono stati forniti dalla prof.essa Rodighiero.\\

La raccolta dati è stata effettuata come illustrato di seguito:
\begin{enumerate}
    \item Posizionare la ghiera in modo che il valore della forza $F$ letto sulla scala sia pari a $200\,g$ (quello effettivamente applicato sarà $4\cdot200\,g=800\,g$).
    \item Azzerare la scala del minimetro posizionando la ghiera in modo che lo zero corrisponda alla posizione della lancetta (che è associata a $F=200\,g$).
    \item Aumentare la forza applicata al filo variando $F$ di $100\,g$ in $100\,g$ fino a $1100\,g$; per ogni valore di $F$ misurare l'allungamento $\Delta x$ indotto sul filo elastico. Si otterranno 10 coppie di valori $(F_i,\,\Delta x_i)$, $i=1,\dots,10$.
    \item Partendo da $F=1100\,g$, diminuire la forza applicata al filo in modo che $F$ vari di $100\,g$ in $100\,g$ sino a tornare al valore di $200\,g$; per ogni valore di $F$ misurare l'accorciamento $\Delta x$ indotto sul filo elastico. Si otterranno 10 coppie di valori $(F_i,\,\Delta x_i)$, $i=1,\dots,10$.
    \item Ripetere i punti 1-4 fino ad esaurimento degli estensimetri a disposizione.
\end{enumerate}   
    
\section{Elaborazione dati}
Di seguito riportiamo la Tabella \ref{tab:elabdat}, mentre nella sezione \ref{sec:tabeldat} i grafici, dei dati elaborati e necessari per la trattazione finale.\\
L'analisi dei dati è stata svolta come illustrato di seguito:
\begin{enumerate}
    \item Definendo il valore iniziale della forza applicata $F_{app,0}=4\cdot F_0$ e con $\Delta F_{app,i}=F_{app,i}-F_{app,0}$ la differenza
        dell'i-esima misura rispetto la forza applicata iniziale, riportare in grafico la dipendenza ($\Delta F_{app,i}\,, \Delta x_i$) nella fase di allungamento e accorciamento del filo. Interpolare i dati relativi alla fase di
        allungamento con una retta di equazione $y=a+b\,x$ e nella fase di accorciamento con una retta del
        tipo $y=c+d\,x$. Riportare i valori di $a,\,b,\,c,\,d$ con i relativi errori dati dall'interpolazione lineare e
        verificare la compatibilità di $a$ e $c$ e la compatibilità di $b$ e $d$ rispettivamente. La costante $K$ del
        filo è data da $b$ e $d$ rispettivamente. Come errore su $x$ utilizzare $1/3$ del valore dato dall'errore
        di sensibilità dello strumento.
    \item Calcolare il modulo di Young $E$ a partire dai valori di $b$ e $d$ così ricavati usando come diametro e
        lunghezza del filo i valori indicati sullo strumento utilizzato.\\
        Ricordiamo che 
        \begin{equation}
            E=\frac{x_0}{SK}=\frac{4\,x_0}{\pi D^2K}
            \label{eq:e}
        \end{equation}
        dove $S$ è la sezione del filo, $K$ la costante di proporzionalità di $\Delta x$ e $F$ e $x_0$ la lunghezza del filo a riposo, calcolare l'errore sul modulo di Young $E$ attraverso la formula di propagazione:    
        \begin{equation}
            \sigma_E=E\sqrt{\biggl(\frac{\sigma_{x_0}}{x_0}\biggr)^2+\biggl(\frac{\sigma_K}{K}\biggr)^2+4\biggl(\frac{\sigma_D}{D}\biggr)^2}
            \label{eq:sigma-e}
        \end{equation}
    \item Calcolare la media pesata dei valori del modulo di Young $E_i\pm\sigma_{E_i}$ ottenuti dalle misure di allungamento e di accorciamento, ricordando che la media pesata è fornita dalla seguente relazione:
        \begin{equation}
            \langle E\rangle=\frac{\sum_i \frac{E_i}{\sigma^2_{E_i}}}{\sum_i \frac{1}{\sigma^2_{E_i}}}\quad\quad\quad dove\quad \sigma_{\langle E\rangle}=\sqrt{\frac{1}{\sum_i \frac{1}{\sigma^2_{E_i}}}}
            \label{eq:media-pesata}
        \end{equation}
    \item \label{itm:k-pes} Calcolare la media pesata $\langle K\rangle$ dei valori $K_i\pm\sigma_{K_i}$ ottenuti dalle misure di allungamento e accorciamento, ricordando che la media è fornita dalle relazione (\ref{eq:media-pesata}), con opportuni aggiustamenti.
    \item \label{itm:e-pes} Verificare la compatibilità della stima $\langle E\rangle$ con $E_{pes}$ ricavata usando la media pesata $\langle K\rangle$:
        \begin{equation}
            E_{pes}=\frac{4\,x_0}{\pi D^2\langle K\rangle}\quad\quad\quad \sigma_{E_{pes}}=E\sqrt{\biggl(\frac{\sigma_{x_0}}{x_0}\biggr)^2+\biggl(\frac{\sigma_{\langle K\rangle}}{\langle K\rangle}\biggr)^2+4\biggl(\frac{\sigma_D}{D}\biggr)^2}
            \label{eq:e-pes}
        \end{equation}
    \item Ripetere i punti 1-5, esaurendo tutti gli estensimetri.
    \item Per gli estensimetri in acciaio verificare la dipendenza lineare di $\langle K\rangle$ rispetto a $1/S$, $S$ sezione del filo, e di $\langle K\rangle$ rispetto $x_0$, $x_0$ lunghezza a riposo dei filo, riportando in grafico %aggiungi ref grafico%
        $\langle K\rangle$ in funzione di $1/S$ e $x_0$.
    \item Verificare che il rapporto $R=\langle K\rangle/x_0$ è costante, utilizzando come valori di $\langle K\rangle$ la media pesata al punto \ref{itm:k-pes}. Riportare in grafico %aggiungi ref grafico%
        il valore $R$ in ordinata rispetto a $x_0$ in ascissa. Associare ad ogni valore di $R$ l'errore dato dalla propagazione
        \begin{equation}
            \sigma_R=R\sqrt{\biggl(\frac{\sigma_{x_0}}{x_0}\biggr)^2+\biggl(\frac{\sigma_{\langle K\rangle}}{\langle K \rangle}\biggr)^2}
            \label{eq:r-pes}
        \end{equation}
    \item Verificare che il prodotto $P=\langle K\rangle\,D^2$ è costante, riportando il valore $P$ in grafico (con i relativi errori) in ordinata rispetto a $D^2$ in ascissa. Associare ad ogni valore di $P$ l'errore dato dall'interpolazione:
        \begin{equation}
            \sigma_P=P\sqrt{4\biggl(\frac{\sigma_D}{x_D}\biggr)^2+\biggl(\frac{\sigma_{\langle K\rangle}}{\langle K\rangle}\biggr)^2}
            \label{eq:p-pes}  
        \end{equation}
\end{enumerate}

\begin{deluxetable}{ccccccccc}[ht]
    \tablecaption{Relazione $\Delta x$ e $\Delta F$}
    
    \tablewidth{0pt}
    \tablehead{
    \colhead{} & \colhead{N°} & \multicolumn{2}{c}{$K$} & \multicolumn{2}{c}{$E$ (\ref{eq:e})} & \colhead{$\langle E\rangle$ (\ref{eq:media-pesata})} & \colhead{$\langle K\rangle$ (\ref{itm:k-pes})} & \colhead{$E_{pes}$ (\ref{itm:e-pes})}\\
    \colhead{} & \colhead{} & \multicolumn{2}{c}{$(10^{-5}\,m/N)$} & \multicolumn{2}{c}{$(10^{10}\,N/m^2)$} & \colhead{$(10^{10}\,N/m^2)$} & \colhead{$(10^{-5}\,m/N)$} & \colhead{$(10^{10}\,N/m^2)$}\\
    \colhead{} & \colhead{} & \colhead{Allung.} & \colhead{Accorc.} & \colhead{Allung.} & \colhead{Accorc.} & \colhead{} & \colhead{} & \colhead{}
    }
    \startdata
    (a) & 1 & 5.26 & 5.29 & 38.7$\pm$0.8 & 38.5$\pm$0.8 & 38.6$\pm$0.6 & 5.28$\pm$0.02 & 38.6$\pm$0.8\,*\\
    (b) & 3 & 4.95 & 3.52 & 11.1$\pm$0.8 & 10.9$\pm$0.8 & 11.0$\pm$0.6\,* & 4.64$\pm$0.23 & 11.0$\pm$0.6\,*\\
    (c) & 6 & 6.39 & 6.38 & 20.4$\pm$0.4 & 20.4$\pm$0.4 & 20.4$\pm$0.3\,* & 6.38$\pm$0.01 & 20.4$\pm$0.4\\
    (d) & 7 & 5.49 & 5.56 & 20.2$\pm$0.4 & 20.0$\pm$0.4 & 20.1$\pm$0.3 & 5.52$\pm$0.02 & 20.1$\pm$0.4\,*\\
    (e) & 10 & 3.53 & 3.53 & 20.8$\pm$0.4 & 20.8$\pm$0.4 & 20.8$\pm$0.3 & 3.53$\pm$0.01 & 20.8$\pm$0.4\,*\\ 
    (f) & 12 & 7.45 & 7.46 & 20.8$\pm$0.4 & 20.8$\pm$0.4 & 20.8$\pm$0.3 & 7.46$\pm$0.01 & 20.8$\pm$0.4\,*\\
    (g) & 14 & 6.39 & 6.35 & 20.5$\pm$0.4 & 20.6$\pm$0.4 & 20.5$\pm$0.3\,* & 6.37$\pm$0.01 & 20.5$\pm$0.4\\
    (h) & 15 & 5.53 & 5.53 & 20.7$\pm$0.4 & 20.7$\pm$0.4 & 20.7$\pm$0.3\,* & 5.53$\pm$0.01 & 20.7$\pm$0.4\\
    (i) & 17 & 3.92 & 3.91 & 20.9$\pm$0.4 & 20.9$\pm$0.4 & 20.9$\pm$0.3 & 3.91$\pm$0.01 & 20.9$\pm$0.4\,*\\
    (j) & 18 & 3.36 & 3.29 & 19.5$\pm$0.4 & 19.9$\pm$0.4 & 19.7$\pm$0.3 & 3.32$\pm$0.01 & 19.7$\pm$0.4\,*\\
    (k) & 19 & 2.38 & 2.42 & 20.6$\pm$0.5 & 20.3$\pm$0.4 & 20.4$\pm$0.3\,* & 2.40$\pm$0.01 & 20.5$\pm$0.4\\
    \enddata
    \tablecomments{La tabella riporta l'elaborazione dei dati al fine di valutare la bontà o meno dell'esperimento. L'analisi di valori è fornita di seguito.}
    \label{tab:elabdat}
\end{deluxetable}

Nella Sezione \ref{sec:tabeldat} sono riportati i grafici dei singoli estensimetri separatamente in allungamento e in accorciamento per permettere una comprensione migliore degli stessi. Si nota senz'altro 
una dipendenza lineare tra i dati interpolati ovvero tra la forza $F$ e l'allungamento/accorciamento $\Delta x$. L'incertezza sulla forza è predominante rispetto all'incertezza 
sull'allungamento/accorciamento, per questo motivo è assimilabile alle dimensioni del punto nel grafico ovvero di $5\cdot10^{-6}\,m$.
Nel grafico dell'estensimetro $3$ si nota sia in allungamento che in accorciamento un punto che si discosta dalla retta di fit; probabilmente è dovuto ad un errore 
sistematico dal momento che è presente in entrambe le misurazioni. Data la numerosa presenza di altre misure e l'impossibilità di ripetere la 
raccolta dati abbiamo deciso di tenere tale errore. Nella Tabella \ref{tab:elabdat} sono riportati per ogni estensimetro i valori di $\langle K\rangle$ con le relative incertezze; 
si noti come l'incertezza su $\langle K\rangle$ dell'estensimetro $3$ si discosti da quelle degli altri estensimetri.
\[
\]
Per calcolare il modulo di Young sono stati utilizzati due metodi:
\begin{enumerate}
    \item Si è calcolato il modulo di Young per ogni estensimetro sia in allungamento che in accorciamento dal coefficiente angolare delle rette interpolanti con la formula (\ref{eq:e}) e poi si è calcolata la media pesata tra il modulo di Young in allungamento e in accorciamento usando la (\ref{eq:media-pesata}).
    \item Si è calcolato il modulo di Young, $E_{pes}$, con la media pesata della costante elastica $K$, $\langle K\rangle$, per ogni estensimetro, utilizzando la (\ref{eq:e-pes}). 
\end{enumerate}
Quindi si sono trovati due valori del modulo di Young per ogni estensimetro. Nella Tabella \ref{tab:elabdat} sono stati segnati i valori più compatibili ai valori forniti dal laboratorio con un asterisco e si nota che la miglior stima
sia proprio $E_{pes}$.
\[
\]
Nella Figura \ref{fig:11a} si è riportato $\langle K\rangle$ rispetto $1/S$ per gli estensimetri in acciaio con lunghezza a riposo costante; mentre
nella Figura \ref{fig:11b} si è riportato in grafico $\langle K\rangle$ e $x_0$ degli estensimetri in acciaio con diametro $D$ costante. 
Come si vede nelle due figure sopra citate si nota una chiara dipendenza lineare tra $\langle K\rangle$ e $x_0$ e un'inversa proporzionalità tra $\langle K\rangle$ e $S$. 
Con questi grafici si verifica che la dipendenza lineare derivata da Young, per ricavare $E$ dalla geometria del materiale sottoposto a tensione, è garantita.
\[
\]
Nella Figura \ref{fig:R} si verifica che il rapporto tra $\langle K\rangle$ e $x_0$ è costante per estensimetri con stesso diametro e lunghezza a riposo diversa; infatti se il rapporto non 
fosse costante non verrebbe conservata la linearità nella Figura \ref{fig:11b}.\\
Nello stesso modo nella Figura \ref{fig:P} si è verificato che il prodotto tra $\langle K\rangle$ e $S$ è costante per gli estensimetri in acciaio con stessa lunghezza a riposo e diametro differente; difatti il prodotto 
deve rimanere costante per mantenere il modulo di Young, che dipende dal materiale cui è composto il corpo sottoposto a tensione, coerente per gli estensimetri analizzati. 








\section{Conclusioni}
Abbiamo verificato la validità della legge di Hooke ovvero la dipendenza lineare tra l'allungamento/accorciamento e la forza applicata su un corpo rigido. 
Abbiamo calcolato con due metodi diversi il modulo di Young $E$ e verificato la loro compatibilità rispetto ai valori noti dei materiali e constatato che in generale 
la miglior stima del modulo di Young sia $E_{pes}$, fornito dalla (\ref{eq:e-pes}).
Infine si è verificato che le dipendenze geometriche rispetto alla sezione e alla lunghezza fossero costanti per $K$.


\newpage
\appendix


\section{Raccolta dati}\label{sec:tabracdat}
  
\begin{deluxetable}{cccccccccc}[h!]
    \tablecaption{Raccolta dati in fase di allungamento e accorciamento}
    \tabletypesize{\scriptsize}
    \tablewidth{0pt} 
    \tablehead{
    \colhead{} & \colhead{} & \multicolumn{2}{c}{Estensimetro 1} & \multicolumn{2}{c}{Estensimetro 3} & \multicolumn{2}{c}{Estensimetro 6} & \multicolumn{2}{c}{Estensimetro 7}\\ 
    \colhead{} & \colhead{} & \colhead{Allung.} & \colhead{Accorc.} & \colhead{Allung.} & \colhead{Accorc.} & \colhead{Allung.} & \colhead{Accorc.} & \colhead{Allung.} & \colhead{Accorc.}\\ 
    \colhead{Forza} & \colhead{Forza} & \colhead{$\Delta x$} & \colhead{$\Delta x$} & \colhead{$\Delta x$} & \colhead{$\Delta x$} & \colhead{$\Delta x$} & \colhead{$\Delta x$} & \colhead{$\Delta x$} & \colhead{$\Delta x$}\\ 
    \colhead{($kg-peso$)} & \colhead{$(N)$} & \colhead{$(10^{-2}\,mm)$} & \colhead{$(10^{-2}\,mm)$} & \colhead{$(10^{-2}\,mm)$} & \colhead{$(10^{-2}\,mm)$} & \colhead{$(10^{-2}\,mm)$} & \colhead{$(10^{-2}\,mm)$} & \colhead{$(10^{-2}\,mm)$} & \colhead{$(10^{-2}\,mm)$} 
    }
   
    \startdata
    0.200$\pm33\%$ & 1.96  & 0.0$\pm0.5$   & 0.0$\pm0.5$   & 0.0$\pm0.5$   & 0.0$\pm0.5$   & 0.0$\pm0.5$   & 0.00$\pm0.5$  & 0.0$\pm0.5$  & 0.0$\pm0.5$ \\
    0.300$\pm33\%$ & 2.94  & 19.0$\pm0.5$  & 19.0$\pm0.5$  & 20.0$\pm0.5$  & 19.0$\pm0.5$  & 25.5$\pm0.5$  & 23.0$\pm0.5$  & 23.0$\pm0.5$  & 19.0$\pm0.5$ \\
    0.400$\pm33\%$ & 3.92  & 40.0$\pm0.5$  & 40.0$\pm0.5$  & 41.0$\pm0.5$  & 39.0$\pm0.5$  & 49.5$\pm0.5$  & 48.0$\pm0.5$  & 45.0$\pm0.5$  & 41.0$\pm0.5$ \\
    0.500$\pm33\%$ & 4.91  & 60.0$\pm0.5$  & 60.0$\pm0.5$  & 60.0$\pm0.5$  & 59.0$\pm0.5$  & 74.5$\pm0.5$  & 73.0$\pm0.5$  & 68.0$\pm0.5$  & 62.0$\pm0.5$ \\
    0.600$\pm33\%$ & 5.89  & 81.0$\pm0.5$  & 81.0$\pm0.5$  & 80.0$\pm0.5$  & 79.0$\pm0.5$  & 100.0$\pm0.5$ & 98.0$\pm0.5$  & 87.0$\pm0.5$  & 84.0$\pm0.5$ \\
    0.700$\pm33\%$ & 6.87  & 102.0$\pm0.5$ & 101.0$\pm0.5$ & 100.0$\pm0.5$ & 100.0$\pm0.5$ & 125.0$\pm0.5$ & 122.5$\pm0.5$ & 108.0$\pm0.5$ & 106.0$\pm0.5$ \\
    0.800$\pm33\%$ & 7.85  & 122.5$\pm0.5$ & 123.0$\pm0.5$ & 120.0$\pm0.5$ & 120.0$\pm0.5$ & 150.0$\pm0.5$ & 148.5$\pm0.5$ & 130.5$\pm0.5$ & 128.5$\pm0.5$ \\
    0.900$\pm33\%$ & 8.83  & 142.1$\pm0.5$ & 145.0$\pm0.5$ & 139.5$\pm0.5$ & 139.5$\pm0.5$ & 175.0$\pm0.5$ & 173.5$\pm0.5$ & 151.5$\pm0.5$ & 150.0$\pm0.5$ \\
    1.000$\pm33\%$ & 9.81  & 164.5$\pm0.5$ & 164.0$\pm0.5$ & 159.0$\pm0.5$ & 160.5$\pm0.5$ & 201.0$\pm0.5$ & 199.0$\pm0.5$ & 173.5$\pm0.5$ & 172.0$\pm0.5$ \\
    1.100$\pm33\%$ & 10.79 & 185.0$\pm0.5$ & 186.0$\pm0.5$ & 179.0$\pm0.5$ & 181.0$\pm0.5$ & 225.0$\pm0.5$ & 224.0$\pm0.5$ & 195.0$\pm0.5$ & 195.0$\pm0.5$ \\
    \enddata    
\end{deluxetable}

\begin{deluxetable}{cccccccc}[h!]
    \tabletypesize{\scriptsize}
    \tablewidth{0pt}
    \tablehead{
    \\\multicolumn{2}{c}{Estensimetro 10} & \multicolumn{2}{c}{Estensimetro 12} & \multicolumn{2}{c}{Estensimetro 14} & \multicolumn{2}{c}{Estensimetro 15}  \\
    \colhead{Allung.} & \colhead{Accorc.} & \colhead{Allung.} & \colhead{Accorc.} & \colhead{Allung.} & \colhead{Accorc.} & \colhead{Allung.} & \colhead{Accorc.}\\
    \colhead{$\Delta x$} & \colhead{$\Delta x$} & \colhead{$\Delta x$} & \colhead{$\Delta x$} & \colhead{$\Delta x$} & \colhead{$\Delta x$} & \colhead{$\Delta x$} & \colhead{$\Delta x$}\\
    \colhead{$(10^{-2}\,mm)$} & \colhead{$(10^{-2}\,mm)$} & \colhead{$(10^{-2}\,mm)$} & \colhead{$(10^{-2}\,mm)$} & \colhead{$(10^{-2}\,mm)$} & \colhead{$(10^{-2}\,mm)$} & \colhead{$(10^{-2}mm)$} & \colhead{$(10^{-2}\,mm)$}
    }
    \startdata
    0.0$\pm0.5$   & 0.0$\pm0.5$   & 0.0$\pm0.5$   & 0.0$\pm0.5$   & 0.0$\pm0.5$   & 0.0$\pm0.5$   & 0.0$\pm0.5$   & 0.0$\pm0.5$ \\
    15.0$\pm0.5$  & 13.5$\pm0.5$  & 28.5$\pm0.5$  & 28.0$\pm0.5$  & 25.0$\pm0.5$  & 35.5$\pm0.5$  & 22.5$\pm0.5$  & 22.0$\pm0.5$ \\
    28.5$\pm0.5$  & 27.5$\pm0.5$  & 58.0$\pm0.5$  & 56.5$\pm0.5$  & 49.0$\pm0.5$  & 51.0$\pm0.5$  & 43.0$\pm0.5$  & 43.0$\pm0.5$ \\
    42.5$\pm0.5$  & 41.0$\pm0.5$  & 87.0$\pm0.5$  & 86.0$\pm0.5$  & 75.0$\pm0.5$  & 77.0$\pm0.5$  & 66.0$\pm0.5$  & 65.5$\pm0.5$ \\
    56.0$\pm0.5$  & 55.0$\pm0.5$  & 117.0$\pm0.5$ & 114.0$\pm0.5$ & 99.0$\pm0.5$  & 100.0$\pm0.5$ & 87.5$\pm0.5$  & 86.5$\pm0.5$ \\
    70.5$\pm0.5$  & 69.1$\pm0.5$  & 146.0$\pm0.5$ & 144.0$\pm0.5$ & 125.0$\pm0.5$ & 125.0$\pm0.5$ & 109.5$\pm0.5$ & 109.0$\pm0.5$ \\
    84.0$\pm0.5$  & 83.0$\pm0.5$  & 175.5$\pm0.5$ & 174.0$\pm0.5$ & 150.5$\pm0.5$ & 150.0$\pm0.5$ & 131.0$\pm0.5$ & 130.5$\pm0.5$ \\
    97.5$\pm0.5$  & 97.0$\pm0.5$  & 204.5$\pm0.5$ & 203.0$\pm0.5$ & 175.0$\pm0.5$ & 176.5$\pm0.5$ & 151.5$\pm0.5$ & 151.5$\pm0.5$ \\
    111.5$\pm0.5$ & 110.5$\pm0.5$ & 233.5$\pm0.5$ & 233.0$\pm0.5$ & 200.5$\pm0.5$ & 200.0$\pm0.5$ & 175.0$\pm0.5$ & 173.5$\pm0.5$ \\
    125.0$\pm0.5$ & 124.5$\pm0.5$ & 262.0$\pm0.5$ & 262.5$\pm0.5$ & 225.0$\pm0.5$ & 224.0$\pm0.5$ & 194.5$\pm0.5$ & 195.0$\pm0.5$ \\
    \enddata
\end{deluxetable}


\begin{deluxetable}{cccccccc}[h!]
    \tabletypesize{\scriptsize}
    \tablewidth{0pt}
    \tablehead{
    \\\multicolumn{2}{c}{Estensimetro 17} & \multicolumn{2}{c}{Estensimetro 18} & \multicolumn{2}{c}{Estensimetro 19}\\
    \colhead{Allung.} & \colhead{Accorc.} & \colhead{Allung.} & \colhead{Accorc.} & \colhead{Allung.} & \colhead{Accorc.}\\
    \colhead{$\Delta x$} & \colhead{$\Delta x$} & \colhead{$\Delta x$} & \colhead{$\Delta x$} & \colhead{$\Delta x$} & \colhead{$\Delta x$}\\
    \colhead{$(10^{-2}\,mm)$} & \colhead{$(10^{-2}\,mm)$} & \colhead{$(10^{-2}\,mm)$} & \colhead{$(10^{-2}\,mm)$} & \colhead{$(10^{-2}\,mm)$} & \colhead{$(10^{-2}\,mm)$}
    }
    \startdata
    0.0$\pm0.5$   & 0.0$\pm0.5$   & 0.0$\pm0.5$   & 0.0$\pm0.5$   & 0.0$\pm0.5$   & 0.0$\pm0.5$ \\
    15.0$\pm0.5$  & 15.0$\pm0.5$  & 13.0$\pm0.5$  & 13.0$\pm0.5$  & 8.0$\pm0.5$   & 9.0$\pm0.5$ \\
    30.0$\pm0.5$  & 31.0$\pm0.5$  & 28.0$\pm0.5$  & 27.0$\pm0.5$  & 17.0$\pm0.5$  & 18.5$\pm0.5$ \\
    45.5$\pm0.5$  & 46.5$\pm0.5$  & 41.0$\pm0.5$  & 39.5$\pm0.5$  & 26.0$\pm0.5$  & 28.5$\pm0.5$ \\
    61.0$\pm0.5$  & 62.5$\pm0.5$  & 53.5$\pm0.5$  & 53.0$\pm0.5$  & 36.0$\pm0.5$  & 37.0$\pm0.5$ \\
    76.5$\pm0.5$  & 77.2$\pm0.5$  & 66.0$\pm0.5$  & 66.0$\pm0.5$  & 46.0$\pm0.5$  & 47.0$\pm0.5$ \\
    91.0$\pm0.5$  & 93.0$\pm0.5$  & 79.5$\pm0.5$  & 78.0$\pm0.5$  & 55.5$\pm0.5$  & 57.0$\pm0.5$ \\
    107.0$\pm0.5$ & 108.0$\pm0.5$ & 93.0$\pm0.5$  & 90.0$\pm0.5$  & 64.0$\pm0.5$  & 66.0$\pm0.5$ \\
    123.0$\pm0.5$ & 122.0$\pm0.5$ & 106.0$\pm0.5$ & 103.0$\pm0.5$ & 73.5$\pm0.5$  & 76.0$\pm0.5$ \\
    138.0$\pm0.5$ & 138.0$\pm0.5$ & 118.5$\pm0.5$ & 117.0$\pm0.5$ & 83.0$\pm0.5$  & 85.0$\pm0.5$ \\
    \enddata
    \tablecomments{Sono riportati i dati nella fase di allungamento e di accorciamento del filo dello strumento.}
\end{deluxetable}

\newpage
\section{Elaborazione dati}\label{sec:tabeldat}
\begin{figure}[h!]
    \centering
        \gridline{\fig{est_1.png}{0.49\textwidth}{(a)}
                  \fig{est_3.png}{0.495\textwidth}{(b)}}
        \gridline{\fig{est_6.png}{0.49\textwidth}{(c)}
                  \fig{est_7.png}{0.495\textwidth}{(d)}}
    \label{fig:graf}
\end{figure}

\begin{figure*}[h!]
    \centering
        \gridline{\fig{est_10.png}{0.49\textwidth}{(e)}
                  \fig{est_12.png}{0.495\textwidth}{(f)}}
        \gridline{\fig{est_14.png}{0.49\textwidth}{(g)}
                  \fig{est_15.png}{0.495\textwidth}{(h)}}
        \gridline{\fig{est_17.png}{0.49\textwidth}{(i)}
                  \fig{est_18.png}{0.495\textwidth}{(j)}}
        \gridline{\fig{est_19.png}{0.49\textwidth}{(k)}}
    %\caption{}
    \label{fig:fit}
\end{figure*}

\newpage

\begin{deluxetable}{ccc}[h!]
    \tablecaption{Valori di $R$ e $P$}
    \tabletypesize{\scriptsize}
    \tablewidth{0pt}
    \tablehead{
    \colhead{Estensimetro} & \colhead{$R$} & \colhead{$P$}\\
    \colhead{} & \colhead{$(10^{-5}$\,$m^2/N$)} & \colhead{$(10^{-12}$\,$m^3/N$)}
    }
    \startdata
    6  & \nodata        & 4.66$\pm$0.09\\  
    7  & \nodata        & 4.76$\pm$0.09\\ 
    10 & \nodata        & 4.57$\pm$0.09\\  
    12 & 7.85$\pm$0.02  & 4.56$\pm$0.09\\  
    14 & 7.96$\pm$0.03  & \nodata\\  
    15 & 7.90$\pm$0.03  & \nodata\\  
    17 & 7.83$\pm$0.04  & \nodata\\  
    18 & 8.30$\pm$0.05  & \nodata\\  
    19 & 7.98$\pm$0.06  & \nodata\\  
    \enddata
    \tablecomments{Sono riportati i valori di $R$ e $P$ e relative incertezze, calcolati rispettivamente con la (\ref{eq:r-pes}) e la (\ref{eq:p-pes})}
    \label{tab:RP}
\end{deluxetable}


\begin{figure}[h!]
    \centering
    \includegraphics[width=0.75\textwidth]{11aa.png}
    \caption{Dipendenza lineare tra $\langle K\rangle$ e $1/S$. Le incertezze sono rispettivamente, per $\langle K\rangle$, quelle riportate nella Tabella \ref{tab:elabdat}, per $1/S$ sono dell'ordine di $10^{-14}\,m^2$ }
    \label{fig:11a}
\end{figure}

\begin{figure}[h!]
    \centering
    \includegraphics[width=0.75\textwidth]{11bb.png}
    \caption{Dipendenza lineare tra $\langle K\rangle$ e $x_0$. Le incertezze sono rispettivamente, per $\langle K\rangle$, quelle riportate nella Tabella \ref{tab:elabdat}, per $x_0$, quelle nella Tabella \ref{tab:filo} }
    \label{fig:11b}
\end{figure}

\begin{figure}[h!]
    \centering
    \includegraphics[width=0.75\textwidth]{fig12.jpg}
    \caption{Costanza del rapporto $\langle K\rangle/x_0$, con $\langle K\rangle$ media pesata dei valori $K$, calcolata al punto (\ref{itm:k-pes}) e $x_0$ lunghezza a riposo del filo, valori riportati nella Tabella \ref{tab:filo}. La media pesata ha un valore di
    $(7.91\pm0.01)\cdot 10^{-5}\,m^2/N$. I valori di $R$ sono riportati nella Tabella \ref{tab:RP}. Si verifica che il rapporto è costante.}
    \label{fig:R}
\end{figure}

\nopagebreak

\begin{figure}[h!]
    \centering
    \includegraphics[width=0.75\textwidth]{13.png}
    \caption{Costanza del prodotto $\langle K\rangle\,S$, con $\langle K\rangle$ media pesata dei valori $K$, calcolata al punto (\ref{itm:k-pes}) e $S$ sezione del filo calcolata con le misure del diametro riportate nella Tabella \ref{tab:filo}. 
    La media pesata ha un valore di $(4.63\pm0.47)\cdot 10^{-12}\,m^3/N$. I valori di $P$ sono riportati nella Tabella \ref{tab:RP}. Si verifica che il prodotto è costante.}
    \label{fig:P}
\end{figure}
\end{document}
